\documentclass[12pt,journal]{IEEEtran}
\begin{document}

\title{Context Aware Authentication Using IoT Services}

\IEEEraisesectionheading{\section{Introduction}\label{sec:introduction}}

The  purpose  of  context  aware  authentication  is  to protect  against  unauthorized  access  to  a  system  in  the event  that  an  attacker  gets  ahold  of  a  user’s  credentials.  Regardless if the credentials were obtained through social engineering  or  compromising  the  system  at  a  technical level,    context    aware    authentication    will    make    it extremely  difficult  for  an  impersonator  to  log  into  the system  by  checking  the  “context”  of  the  surrounding Internet  of  Things  (IoT)  devices  and  determining  if  the intended  user  is  the  person  attempting  to  log  into  the system at a given location.  

Currently, this is only practical for highly secure and localized systems where the user has to login on-premises because the sensor actuator network promised by IoT has not been fully realized. However, once IoT networks are common place and ubiquitous computing becomes a reality, location based access control could potentially become viable for a variety of everyday applications such as webmail, credit card transactions, and accessing property.  The authentication backend of the proposed context aware authentication system put forward in this document will be referred to by its working name, PinPoint.

\section{Security}
\subsection{General Security Concerns}

In any authentication system security is a major concern as authentication is usually one of the major attack vectors that are explored when attackers are attempting to compromise a system. The need for well thought out security measures is only increased in the case of context aware authentication systems because the process for authenticating a user becomes distributed. Now, instead of trying to guess a password or crack a weak password hash, attackers may attempt to impersonate devices or use a man in the middle attack to make it appear that the user is in fact present for login attempt. Furthermore, once context aware authentication systems become widely distributed, a perception that the extra security provided by the system is enough to prevent attacks could potentially cause users to become more careless in choosing strong passwords.

\subsection{Context Aware Security}
The  idea  of  context  aware  security  to  gather  the “context”  of  the  current  login  attempt  from  a  variety  of devices.  These devices could include almost any type of sensor  including  card  scanners,  thermometers,  motion sensors,  and  microphones.    Using  the  context  provided by   these   devices,   PinPoint   would   then   attempt   to determine wheter or not the person attempting to log into the system is indeed the person they are claiming to be.

\end{document}